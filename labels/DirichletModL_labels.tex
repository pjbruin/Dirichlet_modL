%  FILE DirichletModL_labels.tex
%
%
%  Time-stamp: <2024-08-19 15:56:37 john>
%

\documentclass[a4paper, 10pt]{amsart}

\usepackage{amsmath,amsfonts,amsthm,amssymb}
\usepackage{url}
\usepackage{hyperref}
\usepackage{comment}
\newtheorem{thm}{Theorem}[section]
\newtheorem{cor}[thm]{Corollary}
\newtheorem{lem}[thm]{Lemma}
\newtheorem{prop}[thm]{Proposition}
\newtheorem{defn}{Definition}%[section]
\newtheorem{rem}[thm]{Remark}

\hfuzz1pc % Don't bother to report overfull boxes if overage is < 1pc

\def\Z{{\mathbb Z}}
\def\Q{{\mathbb Q}}
\def\C{{\mathbb C}}
\def\F{{\mathbb F}}
\def\Fp{{\mathbb F}_p}
\def\Fq{{\mathbb F}_q}
\def\Fl{{\mathbb F}_{\ell}}
\def\Fld{{\mathbb F}_{\ell^d}}
\def\Flbar{\overline{{\mathbb F}_{\ell}}}
\def\ZNs{(\Z/N\Z)^*}
\def\ZNsh{\widehat{\ZNs}}

\newcommand{\software}[1]{\textsc{#1}{}}
\newcommand{\Sage}{\software{SageMath}}
\newcommand{\Magma}{\software{Magma}}
\newcommand{\Github}{\software{GitHub}}
\newcommand{\GP}{\software{Pari/GP}}


\newcommand{\CLab}[2]{$#1.#2$}
\newcommand{\DLab}[3]{$#1$-$#2.#3$}
\newcommand{\oldDLab}[4]{$#1$-$#2.#3.#4$}

\begin{document}

\title{Dirichlet characters modulo $\ell$ and their labels}

\author{Samuele Anni}
\address{Aix-Marseille Universit\'e, France}
\email{samuele.anni@gmail.com}
\author{Peter Bruin}
\address{Mathematisch Instituut, University of Leiden, Netherlands}
\email{P.J.Bruin@math.leidenuniv.nl}
\author{John Cremona}
\address{Mathematics Institute, University of Warwick, Coventry CV4 7AL, UK}
\email{john.cremona@gmail.com}
\author{Nicolas Mascot}
\address{School of Mathematics, Trinity College, Dublin, Republic of Ireland}
\email{mascotn@tcd.ie}

\date{\today}

\begin {abstract}
We describe a systematic labelling scheme for Dirichlet characters in
prime characteristic $\ell$, which is consistent with the Conrey
labels for classical complex-valued Dirichlet characters.  Each label
has the form \DLab{\ell}{N}{c} where $ell$ is the prime
characteristic, $N$ the modulus, and $c$ an integer coprime
to~$\varphi(N)$ of order coprime to~$\ell$.
\end {abstract}

\maketitle


\section{Introduction}
In \cite{OldDef}, a tentative definition was given for a labelling
system for Dirichlet characters in characteristic~$\ell$, by analogy
with the Conrey labels for classical Dirichlet characters.  That
definition had some inconsistencies, and was rather complicated.

Here we give a new, simpler, definition.  Each character
$\chi:\ZNs\to\Flbar^*$ will have a well-defined label\footnote{The
  exact format of the label is open to adjustment, but will contain
  the same three components $\ell,N,c$; the original definiton had a
  fourth component $d$ which we feel is redundant.  See the last
  section for a discussion.} of the form \DLab{\ell}{N}{c} with
$c\in\ZNs$ such that the map $\chi\mapsto c$ is a group isomorphism
from the group of all such characters to the $\ell$-coprimary
subgroup\footnote{By the $\ell$-coprimary subgroup of a finite or
  torsion abelian group $G$ we mean the subgroup of~$G$ consisting of
  elements whose order is coprime to~$\ell$.} of~$\ZNs$.

We also show how to reduce classical Dirichlet characters and their
labels.  In the simplest case, when the Dirichlet character with label
\CLab{N}{c} has order not divisible by~$\ell$ (which is precisely
when $c\in\ZNs$ has order coprime to~$\ell$), then its reduction
has label \DLab{\ell}{N}{c}.

\section{Notation and preliminaries}

As with Conrey labels, our definition relies on fixing once and for
all isomorphisms between groups which are isomorphic but not
canonically so.  We first set these up and fix some notation.

\begin{itemize}
\item $N$ is a positive integer.
\item $U_N=\ZNs$ is the domain of all characters considered.
\item $X_N=\ZNsh$ is the group of classical Dirichlet characters
  $\chi:U_N\to\C^*$.
\item $c:X_N\to U_N$ is the group isomorphism used in the
  definition of Conrey labels (recalled below), so that $\chi$ has
  label \CLab{N}{c(\chi)}.
\item $\ell$ is a prime number.
\item for each $d\ge1$, $f_{\ell,d}(X) \in \Fl[X]$ is the Conway
  polynomial of degree $d$ modulo~$\ell$ as defined by R.~Parker (see
  \cite{ConwayPol}); the properties of these will be recalled below.
\item for each $d\ge1$, $\Fld=\Fl[X]/(f_{\ell,d})$ is the specific
  model for the extension of $\Fl$ of degree $d$ defined by the Conway
  polynomial of degree $d$, with the associated structure as
  $\Fl[X]$-algebra (and not just as $\Fl$-algebra); $z_d\in\Fld^*$ is
  the image of $X$ in $\Fld=\Fl[X]/(f_{\ell,d})$; it is a generator of
  the cyclic group~$\Fld^*$, and $\Fld=\Fl[z_d]$.  Properties of
  Conway polynomials mean that we may embed
  $\Fld\hookrightarrow\F_{\ell^{de}}$ via $z_d\mapsto
  z_{de}^{(\ell^{de}-1)/(\ell^d-1)}$.
\item For $r\in\Q$, define $e(r)=\exp(2\pi ir)$, so that $e$ induces an
  isomorphism from~$\Q/\Z$ to the group of all complex roots of unity.
\item for each $d\ge1$ we fix the isomorphism
  $\Fld^*\cong\mu_{\ell^d-1}(\C)$, the group of complex roots of unity
  of order dividing $\ell^d-1$, mapping $z_d$ to $e(1/(\ell^d-1))$.
\item taking the direct limit over all $d\ge1$ gives a fixed
  isomorphism $\Flbar^*\cong\mu^{(\ell)}(\C)$, where
  $\mu^{(\ell)}(\C)$ denotes the group of complex roots of unity of
  order coprime to $\ell$; both these groups are isomorphic to the
  $\ell$-coprimary subgroup of $\Q/\Z$, and hence to
  $\oplus_{p\not=\ell}(\Q_p/\Z_p)$.
\end{itemize}

\subsection{Conrey labels}
We briefly recall the definition of Conrey labels, which relies on
defining a specific isomorphism $X_N\to U_N$.  See
\cite{ConreyLabels} for more details. By the Chinese Remainder Theorem
it suffices to consider the case where $N$ is a prime power, $N=p^e$.


For $p$ odd the group $ U_N$ is cyclic.  Fix $g_p$ to be the smallest
positive integer generating $U_{p^e}$ for all~$e\ge1$.  For
$\chi\in X_N$, define $c(\chi)\in U_N$ by $\chi(g_p) =
e(c(\chi)/\varphi(N))$.

For $N=2^e$ with $e\ge3$ (the cases $e=1,2$ being trivial) we note
that $ U_N$ is generated by $-1$ and~$5$, and define
$c(\chi)=\chi(-1)c_1$ where $\chi(5) = e(c_1(\chi)/\varphi(N))$.

\subsection{Conway polynomials}
We do not define the Conway polynomials $f_{\ell,d}$ here but refer to
\cite{ConwayPol}, and recall their relevant properties:
\begin{enumerate}
\item Each $f_{\ell,d}\in\Fl[X]$ is \emph{irreducible} of degree~$d$.
\item Each $f_{\ell,d}$ is \emph{primitive}, meaning that its roots
  in~$\Fld$ have order $\ell^d-1$ and so generate the multiplicative
  group~$\Fld^*$.
\item The $f_{\ell,d}$ for $d\ge1$, ordered by divisibility, are
  \emph{coherent}, meaning that if $z$ is a root of $f_{\ell,de}$ for
  $e\ge1$, then $z^{(\ell^{de}-1)/(\ell^d-1)}$ is a root
  of~$f_{\ell,d}$.
\end{enumerate}
Using the coherence property, if for each~$d\ge1$ we define
$z_d\in\Fld^*$ to be the image of $X$ in the model $\Fld =
\Fl[X]/(f_{\ell,d})$, then:
\begin{enumerate}
\item each $z_d$ generates the multiplicative group $\Fld^*$;
\item $z_{de}^{(\ell^{de}-1)/(\ell^d-1)}=z_d$, in the sense that the
  inclusion $\Fld\hookrightarrow\F_{l^{de}}$ is induced by $X\mapsto
  X^{(\ell^{de}-1)/(\ell^d-1)}$ and so sends $z_d$
  to~$z_{de}^{(\ell^{de}-1)/(\ell^d-1)}$;
\item this coherent choice of generators $z_d$ defines isomorphisms
  $\Fld^*\to\mu_{\ell^d-1}(\C)$ via $z_d\mapsto e(1/(\ell^d-1))$.
\end{enumerate}

\section{Definition of Dirichlet character modulo \texorpdfstring{$\ell$}{ell}}

\begin{defn} A \emph{Dirichlet character modulo~$\ell$}, or a
  \emph{mod-$\ell$ Dirichlet character}, with modulus $N$, is a group
  homomorphism
  \[
  \chi:   U_N \to \Flbar^*.
  \]
\end{defn}
Note that in this definition, the codomain of a mod-$\ell$ character
is~$\Flbar^*$ rather than the multiplicative group of a specific
finite field~$\Fld$.  Since $ U_N$ is finite of order $\varphi(N)$,
each character has finite image, and the images of all characters (for
fixed~$N$) will lie in $\Fld$ for $d$ satisfying
$\ell^d\equiv1\pmod{m}$, where $m$ is the coprime-to-$\ell$ part
of~$\varphi(N)$ (or even of the exponent of the group~$ U_N$, rather
than its order).  This definition is more convenient than allowing
different characters to have different codomains $\Fld^*$ for
different~$d$.  In particular, if a character is defined by a
homomorphism $ U_N\to\Fld^*$ for some specific $d$, and then the
codomain is extended to~$\F_{l^{de}}^*$, we will not regard the
resulting composite as a different character, and its label will not
change.

Since $\Flbar^*$ has no $\ell$-torsion, the order~$m$ of every
mod-$\ell$ character~$\chi$ is coprime to~$\ell$, and the values
of~$\chi$ lie in $\Fld^*$, where $d$ is the multiplicative order
of~$\ell$ modulo~$m$.

\subsection{Lifting characters and the definition of labels}

Let $\chi:  U_N \to \Flbar^*$ be a mod-$\ell$ Dirichlet character
of order~$m$, and let $d\ge1$ be the multiplicative order of $\ell$
modulo~$m$, so that $m\mid\ell^d-1$ and the values of~$\chi$ lie in
$\Fld^*$.  Composing with our fixed isomorphism $\Fld^* \to
\mu_{\ell^d-1}(\C)$, mapping $z_d\mapsto e(1/(\ell^d-1))$, gives a
classical Dirichlet character $ U_N\to\C^*$, which we denote
$\tilde{\chi}$.  We say that $\tilde{\chi}$ is the \emph{lift} of
$\chi$ to characteristic zero.

Because of our choice of Conway models for finite fields and the
coherence property of the generators $z_d$, the definition of
$\tilde{\chi}$ is unchanged if we replace $\Fld^*$ by any extension
of~$\Fl$ containing the values of~$\chi$.

We define the label of $\chi$ to be \DLab{\ell}{N}{c}, where
\CLab{N}{c} is the label of~$\tilde{\chi}$.

\section{Reduction of characters and their labels}

Now let $\chi\in X_N$ be a classical Dirichlet character, of
order~$m$ and label \CLab{N}{c}.  By definition of Conrey labels, $m$
is also the multiplicative order of $c\in U_N$.  We will define a
mod-$\ell$ Dirichlet character (with the same modulus $N$), called the
reduction of $\chi$ modulo~$\ell$, and denoted $\overline{\chi}$, and
will see how to determine its label from that of~$\chi$.

The definition needs to take into account the fact that~$m$, the order
of $\chi$, may not be coprime to $\ell$.  Such characters exist, with
modulus $N$, if\footnote{In \cite{OldDef} there seems to be an
  assumption that this can occur if and only if $\ell\mid N$, which is
  incorrect.} and only if $\ell\mid\varphi(N)$, which is if and only
if either $\ell^2\mid N$ or some prime $p\equiv1\pmod{\ell}$
divides~$N$.

\begin{itemize}
  \item[Special case:] Assume that $\ell\nmid m$. Let $d$ be the
    multiplicative order of $\ell$ modulo~$m$, so that $\Fld^*$ has a
    subgroup of order~$m$, generated by $z_d^{(\ell^d-1)/m}$.  Using
    our fixed isomorphisms we have an injection
    $\mu_m(\C)\hookrightarrow\Fld^*\hookrightarrow\Flbar^*$, mapping
    $e(1/m)\mapsto z_d^{(\ell^d-1)/m}$.  Composing $\chi$ with this
    injection we obtain a mod-$\ell$ character which we
    denote~$\overline{\chi}$, of the same order as~$\chi$, and such
    that $\tilde{\overline{\chi}}=\chi$.  Hence the label of
    $\overline{\chi}$ is \DLab{\ell}{N}{c}, with the same value
    $c\in U_N$ as in the label \CLab{N}{c} of the original character
    $\chi$.

    Note that the definition of $\overline{\chi}$ does not depend on
    which extension $\Fld$ of $\Fl$ we use here, provided that it
    contains the $m$th roots of unity, that is provided that
    $\ell^d\equiv1\mod{m}$.

    \item[General case:] In general, write $m=m_1\ell^k$ where
      $\ell\nmid m_1$ and $k\ge0$.  Then $\chi$ has a unique
      factorization as $\chi=\chi_1\chi_{\ell}$ where $\chi_1$ has
      order~$m_1$ and $\chi_{\ell}$ has order~$\ell^k$.  Explicitly,
      let $u,v$ be integers such that $1=m_1u+\ell^kv$; then
      $\chi_1=\chi^{\ell^kv}$ and $\chi_{\ell}=\chi^{m_1u}$, both
      being independent of the choice of $u,v$.  (When $k=0$ we may
      take $u=0$, $v=1$, so that $\chi_1=\chi$; this is the special
      case already handled.)

      Now we define $\overline{\chi}=\overline{\chi_1}$, the latter
      being defined as in the special case.  The label of $\chi_1$ is
      \CLab{N}{c_1}, where $c_1\equiv c^{\ell^kv}\pmod{N}$, since the
      map from Dirichlet characters to the second component of their
      Conrey label is an isomorphism $X_N\to U_N$.  Hence the label
      of $\chi$ is \DLab{\ell}{N}{c_1}.
\end{itemize}

\section{Additional remarks}
\begin{enumerate}
\item
  The set of all mod-$\ell$ Dirichlet characters with a fixed modulus
  $N$ forms a group.  The map from a character with label
  \DLab{\ell}{N}{c} to the last component~$c$ is an isomorphism from
  this group to the $\ell$-coprimary subgroup of $ U_N$.
\item
  The reduction map is a group homomorphism from $X_N$ to the group
  of mod-$\ell$ Dirichlet characters with modulus $N$.  It takes the
  character with label \CLab{N}{c} to \DLab{\ell}{N}{c_1}, where
  $c\mapsto c_1$ is the projection from $ U_N$ to its $\ell$-coprimary
  subgroup.  This map is only injective when $\ell\nmid\varphi(N)$.
  In general, the reductions of \CLab{N}{c} and \CLab{N}{c'} are the
  same if and only if $c/c'$ has order a power of $\ell$.
\item
  Lifting gives a splitting to the reduction map, and is an
  isomorphism from the group of all mod-$\ell$ Dirichlet characters to
  the $\ell$-coprimary subgroup of $X_N$. In terms of labels it
  takes \DLab{\ell}{N}{c} to \CLab{N}{c}.
\item
  The original definition regarded a mod-$\ell$ character to be a map
  $ U_N\to\Fld^*$ for a specific~$d$, and included $d$ in the label:
  instead of \DLab{\ell}{N}{c}, the label was \oldDLab{\ell}{d}{N}{c}.

  We think that including $d$ in the label is unnecessary, and even
  confusing, since the character \oldDLab{\ell}{d}{N}{c} is unchanged
  by replacing $d$ by any multiple of~$d$.  Similarly, the Conrey
  label \CLab{N}{c} could, but does not, include a component $m$,
  to mean that the character values lie in a specific cyclotomic field
  $\Q(\zeta_m)$.

\item
  Implementation notes: both \Sage\ and \Magma\ use Conway polynomials
  by default in constructing finite fields.  It appears that \GP\ does
  not, though the user can specify which polynomials to use via the
  function \texttt{ffgen()}.
\end{enumerate}

\begin{thebibliography}{99}
\bibitem{OldDef} Original version of labels for mod-$\ell$ Dirichlet
  characters:
  \url{https://github.com/sanni85/Dirichlet_modL/blob/master/LABELS_old.md}
\bibitem{ConreyLabels}
  Definition of Conrey labels for mod-$\ell$ Dirichlet characters:
  \url{https://www.lmfdb.org/knowledge/show/character.dirichlet.conrey}
\bibitem{ConwayPol} Definition of Conway polynomials:
  \url{http://www.math.rwth-aachen.de/~Frank.Luebeck/data/ConwayPol/index.html}.
\end{thebibliography}

\end{document}

